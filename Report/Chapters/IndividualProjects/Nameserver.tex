\section{Nameserver}\label{s:nameserver}
The nameserver interface is organized in a fairly straightforward manner.
To register, it expects you to fill out an nameserver\_info struct and give it to register\_service.
The nameserver\_info struct requires you to give it a unique name which does not contain the letter ','. Further you also need to provide it a type - which can be any user chosen string that is not ','. (in both cases whitespace in the strings is fine) Which means querying on types is a good start for getting services.
Further, you need to give it the current coreid, this is of little use at the moment as we will see later, but exists for a possibl extension of the nameserver such that existing services do not need to be rewritten nor any adjustments to the nameserver interface is needed later.
Furthermore you need to provide it with an capability which is the local cap of an open channel that accepts new connections.
And then, finally, you can provide 0 to n properties, which are of the form "key:value" as they present a full on key value store. (it also means that neither keys are not allowed to contain ':', further, neither keys nor values may contain '{' or '}' but whitespace is fine)
All this will then get registered with the nameserver on the register\_service call - provided the name has not been used before.
The next important concept is the nameserver\_query, this implements a reasonably powerful query language (though not as powerful as desired - see attached listing for originally intended query language compared to cut down version). One can query on name identity, type identity and a set of properties existing with certain keys and the values comparing to the provided values in specific ways.
Said different: Every nameserver\_query can have N property query objects, which state which key they want to look for (not finding it counts as nonfullfillment. Repeat mentions of the same key are legal) and then can also specify a string to compare the stored value to - either by string equality, the stored value starting with the provided, ending with it or containing it.
Since this can be decided for every single query object and we can have an essentially unrestricted amount of those we can write fairly complicated queries on those properties.
Sadly, I ran out of time to properly integrate name and type into this system, in particular since other parts of the system already went gold and changing this would have constituted a breaking change for them and required a full retesting. However, they definitely should be integrated such that an uniform querying is possible (likely by having them be real or pseudo properties of the kind (name,<provided name>) and (type,<provided type>)
Once we have a nameserver\_query we can then send it to lookup to get the first fit or enumerate to get the list of names of all fits.
On a sidenote, by simply asking to query for props and leaving the list of query prop empty, every registered service fullfills all of those trivial and enumerate returns the full list of registered services
since lookup returns a nameserver\_info struct with all the data the original process that registered it provided, we can then do arbitrary processing on it afterwards if we feel like the provided query system is not powerful enough - though we can not write back the change, as will be explained later - thus protecting the system from some attempts at fraudulent changes.
further, the names provided by enumerate, since they are unique, are clear identifiers of the registered services and we can just send out lookup calls for all of them (or some) if we desire their actual nameserver\_info
once we have received a nameserver\_info struct from the nameserver that we are satisfied with, we can then take the cap we were given there and establish a new connection with the process providing it. In an interesting side note, we do not know the true identity of the process nor do we care, as our current setup of the rpc\_system is that we just establish a completly fresh channel with 2 freshly made caps on both sides (1 per side) to talk with each other.
As our memory manager, process server and serial console were never moved out of init, we just have init register all those services and the clients who get the caps to connect to those services never know on what process it actually is running, so moving those out would be an entirely transparent change from anything other than the initial set's view. Sadly, the time to move them out just wasn't there. I'll elaborate later on how it could be done
On another sidenote, since we get all the info the original process had given to registrate, we could now use that cap to register a new service which leads to its endpoint and may be used to lead services astray. This, however, creates no additional security risks since:
1. if having another process send the wrong kind of things to the service we got the cap from endangers the service, we could just have send those messages directly
2. any system can register any service currently anyway and even if it holds the end of the line for that, that can break things.
3. Just not initializing the channel the cap refers to (or sending an unrelated cap) could lead to an unsuspecting other service crashing the kernel by trying to send on that
4. we don't even have to jump through any of those loops, just create a particularly constructed bad channel and try to send on it and thus crash the kernel on our own, without having to involve the nameserver

What we do have, in regards to protection, however, is that a process can only deregister service that it itself has registered, this works by giving the service name to the deregister call of the nameserver. On a sidenote, the ability to register and deregister together means we do not need a call to be allowed to change properties, this can simply be done by either still having the original nameserver\_info, or looking it up in the nameserver, modifying it as desired, deregistering the original and registering the changed under the same name. Since all established channels are directly with our process and not over the nameserver, this does not impact established services. Further, since we can just send the same cap we already registered there, to anyone not querying in precisely the small window it was deregistered it looks like a modify call occured.
In the future it might be worth considering to make register\_service be allowed to overwrite if the original service with that name came from the same process. However this could also lead to accidental overwriting so it would have to be weighed carefully, however it seems like a worthwhile feature to me if we had the time to test (as it technically constitutes a breaking change in the general system)
there we also provide a nice convenience function with remove\_self, which deregisters all services registered by this process.
As for how the checking for identity works, it's actually pretty simple:
The first call to the nameserver first gets the nameserver capability from init, then does a handshake with the nameserver where a new exclusive channel gets created between only this process and the nameserver. Thus the nameserver simply identifies the process by anything that comes over that channel. This has the nice side effect that simply by handing the cap to send there away it should - and this is untested as we had no use for it and it is slightly hacky - allow a process to give other processes the right to register and deregister processes in its name.
We do not have a hierarchical naming structure, as having a rich property system not only is more useful than that - ad hoc category creation and being in arbitrary subsets of them, especially once we consider the different ways we can query on - we can, if desired, also simulate a hierarchical naming structure with props, simply by adding a lvl0 prop, a lvl1 prop, etc etc which can then be understood to represent such a hierarchical tree.
sadly, automatic dead service removal does not quite work yet - services need to manually request their removal. While an attempt was made, the idea of looking for a lmp\_chan changing to disconnect state as a witness of the other side being dead wasn't fruitful and then once more time constraints prevented further inquiry and approaching the topic. It would be relatively simple to have the process server notify us or something along those lines, though, as that one can successfully track processes not running anymore.
As for bootstrapping and the initial set - since we have the memory manager and the process manager still integrated into init, these just boot first and the nameserver gets its caps from there. Now, we could - with some work - make it so that init boots, then the memserver which then stays in a spinloop with init waiting for the nameserver to appear, then the nameserver could appear, register its cap with init, then wait for the memserver to register with it, immediately do a handshake with it once it does, then the memory server could keep requesting on the nameserver for the proc manager, which init now boots up, that thing gets the nameserver cap, gets the memserver cap from the name server (or waits until both of those are available if it boots too quickly), then could register itself with the nameserver and then the nameserver and the memserver could register themselves with the proc manager and all three come fully online at essentially the same time, with only the nameserver cap having ever had to go to init. This would be one way - the absolutely minimum "going over init" way of solving it.
Another simpler way would be to have an initial set of processes which just work like this set would have worked without the existance of a nameserver (this likely would at least have to be init+memserver+nameserver, if not also the procmanager, depending on implementation details) and then the namserver just gets added either after memserver boot or after proc manager boot.
Back to our actual system - once the nameserver is up, all other processes do get their memserver, proc manager and serial console caps from the nameserver. Further, the network system uses the nameserver to allow its various parts to establish intercommunication
However, one core limitation of the nameserver as it is currently - primarily due to time reasons and us never having constructed an unified RPC and URPC system - core specific. That is, it can only provide connections to other services running on the same core. As our system is mostly a bipartited system with most functionality replicated across cores to some degree or another, this is actually somewhat helpful, as it means every process will be guaranteed to get the correct memory manager.
However, if time had permitted it, one simple way to improve on this would have been to extend the nameserver and the proxy server in such a way that the nameserver replicates service entries across cores unless they are marked as core exclusive by using an extended proxy server which is extended in such a way as to allow us to fake a lmp\_chan across URPC, by opening a lmp\_chan on the other side, being able to deal with chan spawning via handshakes and just tunneling all data over URPC. Then the nameserver on the core the process was not originally from, instead of having the actual process's chan capability in its entry, has one generated by the proxy server which is set up such that it will direct any calls to the original process.
This is a bit tricky to make work with proper handshake propagation (as it spawns new channels), but should still be doable as the proxy server initiates it on one end and knows where it came from on the other end.
However, sadly this did not make it past the planning stage due to time.