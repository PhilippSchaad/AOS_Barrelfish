\section{TurtleBack (Shell)}\label{s:turtle-back-shell}

The shell is a very essential part of any OS, as it allows the user to interact 
with the system at its most basic level. However, for the shell to make any 
sense to begin with, one first needs a way to talk with the user. This is 
typically done with a standard text interface. Our system thus needs to have 
the ability of printing text to the screen, and read text back that has been 
typed in by the user. It is best to do this reading and writing as centralized 
as possible, and for this we will first look at the subsystem responsible for 
that bidirectional input and output; the terminal driver.

\subsection{Terminal Driver}\label{ss:terminal-driver}

Our terminal driver is kept as simple and protocol free as possible. For this 
reason it is running in the \texttt{init} domain on both cores. Let's 
first discuss what this implies for our output to the screen.
\medskip

When printing to the screen with functions from the C standard library, the 
system first checks whether we are in the \texttt{init} domain or not. If we 
are indeed, we directly talk to the terminal driver, requesting to have it 
print to the screen for us. The terminal driver (since it is inside of the 
\texttt{init} domain) can do so via system call. If we are however in a 
different domain, we will send out an RPC to \texttt{init}, which in turn uses 
the terminal driver to service our request and print to the console via system 
call. This can happen on both cores simultaneously and entirely independently, 
which keeps the system's complexity at a minimum.
\medskip

Input is a little less trivial, but still very simple. Since we have two 
instances of our terminal driver running (one on each core), we will call the 
one on core 0 our master and the one on core 1 our slave instance. Since we can 
only have at most one instance registered for interrupts from the UART 
interface, we will let the master instance take care of interrupt handling, and 
hence register it to receive UART interrupts. If an interrupt is received by 
the master instance, it will perform a system call to retrieve the typed in 
character from the UART buffer and store it in its own input buffer (which we 
will get to in a second). Since the slave instance will not receive that 
interrupt and does not have the ability to retrieve that same character from 
the UART again, the master instance performs a URPC with which it sends the 
received character to the slave instance, which can add it to its own buffer.

\subsubsection{Input Buffers}\label{sss:input-buffers}

Our terminal driver supports two separate modes of buffering input from the 
UART interface. Line- and direct-mode.
\medskip

In line-mode we buffer - as the name suggests - entire lines of text. Our line 
buffer consists of two separate character arrays, one of which is considered to 
be our write-buffer and the other one is the read-buffer. When a new character 
arrives (via interrupt if we are on core 0, via URPC otherwise), we will first 
perform a little bit of preprocessing with it. The most obvious step is to 
check if it is a character or signal indicating the end of a line (EOT, LF, 
CR). In that case we null terminate our write-buffer and switch the two 
buffers. So now when someone tries to read from the read-buffer, it will get a 
character from the newly finished line, and we will start writing to a new (now 
invalid and thus considered to be empty) line. The remaining preprocessing 
steps are less important and mainly matter for the second point where the two 
modes differ from each other; echoing. In line-mode let the terminal driver 
manage echoing of newly arriving characters, and not whatever process is 
reading that input. This is also why we do not use this mode for the shell, but 
instead use our second mode of operation.
\medskip

In direct-mode the steps performed by the terminal driver are kept to a bare 
minimum. Our buffer has the form of a ring-buffer (FIFO), and whenever a new 
character arrives, we simply add it to the buffer and advance the tail by one, 
without really caring about what the character value was. This gives the mode 
its name, as we will directly push ALL input forward to the consuming 
process(es), without performing any cleanup or echoing the characters back to 
the user. Our consumer has to take care of that.
\medskip

In both of those modes we simply discard new characters if our buffer is 
already full. In addition to that, escape and control sequences that make a 
mess or might be important for the consumer are analyzed by the terminal driver 
and stored as one special character (instead of their raw form, which typically 
is 2-3 characters). This can be easily done, since the ASCII standard for our 
supported symbols only consumes 128 values. A character (byte) can hold 256 
values, so we can use the remaining spaces to encode 128 custom characters. For 
example left-, up-, right-, and down-arrow keys will get escaped by the 
terminal driver and stored as special values, so the consuming process 
understands that they were pressed.
\medskip

When a process performs an input operation using one of the C standard library 
functions, we differentiate between \texttt{init} domain and other domains 
again. In the case of \texttt{init}, we directly query the terminal driver for 
new characters. Other domains will perform an RPC to init, which will serve the 
character via the terminal driver. In both cases, the call to the terminal 
driver will block for as long as the input-buffer does not have any ready 
characters. As soon as characters arrive, one is returned and the consumer 
process can analyze it. Which input-buffer will get queried depends on the 
currently selected mode in the terminal driver. To make sure the two driver 
instances' input-buffers are in sync, whenever a character is removed from one 
of the two instances, it will fire a URPC to the other instance, moving the 
head by one space, thus also removing the character.

\subsection{The Shell}\label{ss:the-shell}

Our shell is a separate process (\texttt{turtleback}) that can be spawned on 
either core once the system has completed its boot process (core 0 by default).
The main shell process in itself is not very interesting. All its doing is 
performing an endless loop of reading characters from the terminal (via C 
standard library functions, which go over \texttt{init}'s terminal driver), 
storing them in a buffer and then parsing that line once the user hits a new 
line. The interesting part lies in just the parsing of that input line and what 
that sets in motion.
\medskip

The shell serves two main functionalities. It presents the user with a set of 
built in functions - with the unspectacular name 'builtins' - that can be used 
to interact with the operating system, and it will spawn other processes for 
the user when he types their name into the console. Let us first examine the 
built in functions that our shell provides.

\subsubsection{Builtins}\label{sss:shell-builtins}

\begin{itemize}
	\item \texttt{help} - This is probably the most basic builtin that any 
	shell (or application for that matter) should have. It does exactly what 
	the name says; it prints out a list of all available builtin functions and 
	offer a helpful little explanation of what each of them does. Supplying it 
	with the name of another builtin will print more information about how to 
	use said function.
	
	\item \texttt{clear} - Another relatively self explanatory function, found 
	in pretty much every command-line based interface; it will clear the screen 
	for you (preserving the scroll buffer). This is done by printing the ANSI 
	sequence ESC - {[} - 2J to the terminal.
	
	\item \texttt{echo} - Also one of the functions everyone knows; this will 
	simply print whatever arguments you supply it with back out onto a new line 
	in the terminal. It will print all 'words' (arguments) spaced out with one 
	single space, no matter how far apart the arguments were supplied (yes, it 
	behaves exactly like bash's \texttt{echo}).
	
	\item \texttt{ps} - Like its bash equivalent, \texttt{ps} will simply 
	perform an RPC to the process manager (in the init domain), which in turn 
	will print a list of all running processes, together with their process ID 
	and which core they are running on.
	
	\item \texttt{time} - This is the last one of the commands that behaves 
	pretty much like bash; \texttt{time} must be used as a prefix command to 
	any other builtin or binary (with arguments or not). It simply times how 
	long it takes for the execution of said command/program to complete. This 
	is done by getting the clock-cycle count before and after the execution, 
	calculating the difference and dividing it by the CPU clock frequency, 
	which gives us the exact time in seconds and milliseconds.
	
	\item \texttt{led} - A new, but rather unspectacular command, that simply 
	tells the \texttt{init} domain to toggle the D2 LED via RPC. If the LED was 
	on it gets turned off, and vice versa. Toggle. That's all. Yes.
	
	\item \texttt{testsuite} - This performs a little RPC to the \texttt{init} 
	domain, which will run a standard test suite that is built in to the OS and 
	contains a couple of memory and regression tests that were used during the 
	development of this operating system. It only exists because we can.
	
	\item \texttt{memtest} - The memory test will take a single argument (a 
	positive number), that tells it how many bytes the user wants to test. It 
	will then allocate a contiguous region of memory of that size and test it 
	for read- and write-errors.
	
	\item \texttt{threads} - This is simply a demo of user level threads. It 
	takes a single positive integer argument and will spawn a number of threads 
	equal to that integer argument. Those threads simply print out the line 
	\texttt{"Hello world from thread X"}, where \texttt{X} is its thread-id.
	
	\item \texttt{oncore} - Serving as another one of those prefix commands 
	(like \texttt{time}), \texttt{oncore} will take the program spawning 
	command it prefixes and spawn said binary on the core specified. So if we 
	type in \texttt{'oncore 1 hello args..'}, this will spawn the binary 
	\texttt{'hello'} on core 1 with arguments \texttt{'args..'}. This is done 
	with a standard spawn-process RPC to init, which can already handle 
	cross-core spawns.
	
	\item \texttt{detached} - This is the final prefix command like 
	\texttt{time}. Like \texttt{oncore}, it serves as a prefix purely for 
	spawning programs, not shell builtins. Its purpose is pretty simple; 
	\texttt{'detached progname args..'} spawns the program \texttt{'progname'} 
	with arguments \texttt{'args..'}, but will not wait for it to finish 
	before returning to a new shell prompt and listening for user input.
\end{itemize}

\subsubsection{Parsing User Input}\label{sss:parsing-input}

We parse the user's input every time he or she hits the end of a line and 
starts a new one (typically by pressing \texttt{Ctrl + J}, \texttt{Ctrl + D}, 
or \texttt{Return}). The line of text in the buffer gets split up into a series 
of token. Each space in the line represents the end of a token and the start of 
the next one, whereby multiple consecutive spaces get dropped and will count as 
a single token separator. We now look at the first token and treat it as our 
command, thus deciding what to do based on its value.
\medskip

We first check if the user wants to execute one of the builtins TurtleBack 
provides. Our shell holds an internal list of all available built-in functions, 
together with a handler function attached to each one of them. We can thus 
simply iterate over this (relatively short) list, performing string matching 
against each of the identifying commands.
If we have a positive match, we will execute that match's handler function, 
passing all but the first token along as its arguments.
If we hit the end of the list without a match, we move on, treating the entered 
command as a binary the user is trying to spawn.
\medskip

This leads us to how we handle process spawning. We use an UNIX-like approach, 
where a program can be launched simply by typing the name of its binary into 
the console, without a special run command. Internally this will launch an RPC 
to the \texttt{init} process, where we look up if the queried binary name can 
be found in our multiboot image (since we do not support a file-system, that's 
the only place we can spawn from). \texttt{init} then launches the process 
(again, passing all the remaining tokens along as arguments to the new 
program), registering it in our process manager, and returns the PID it just 
assigned to the process. TurtleBack then queries \texttt{init} (via RPC) for 
the existence of that process, blocking until that PID could not be found 
anymore. This ensures that we await the termination of our spawned program 
before showing a new shell prompt and accepting user input, thus allowing the 
spawned process to take control.
\medskip

If the requested binary could not be found, the PID returned will have a 
special value (UINT32\_MAX), which tells TurtleBack that we still have not been 
able to handle the user's command. At this point we know that if we were unable 
to serve the user's request, he or she must have entered a wrong command, so we 
echo the command back (just the first token) and tell the user that this does 
not match to any supported command.