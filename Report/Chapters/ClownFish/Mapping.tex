\section{Memory Mapping}\label{s:mapping}

\subsection{Structures and Bookkeeping}\label{ss:mapping_structures}

To keep track of the various mapped regions and pieces of memory, as well as 
the free virtual addresses, we have to keep a significant amount of state for 
the memory mapping and paging process.

\begin{itemize}
	\item Free vspace: To prevent address collisions, the free virtual 
    addresses are tracked. This is done using a singly linked list, because 
    its simplicity. This might introduce additional overhead when mapping 
    memory, when the process has run some time and the memory is fragmented. 
    The lookup scales with $0\klammern{n}$. Currently, freed nodes are simply 
    appended to the list and scale with $0\klammern{1}$ 

	\item Mappings: To be able to unmap a piece of memory again, the mappings 
    are stored. This is done using a singly linked list. A new mapping is 
    added in front of the list. This scales with $0\klammern{1}$. Finding the 
    right mapping for unmapping memory scales with $0\klammern{n}$ with the 
    number of mappings in the worst case, when it has to traverse the whole 
    list. However, we think that there are two sorts of mappings. One is short 
    lived and will be unmapped soon after it is mapped. And the others are 
    long living. Our implementation let the long living mappings wander to the 
    back of the list while being fast for unmapping recently added mappings.

	\item L1 pagetable: a reference to the L1 pagetable is stored.

	\item L2 pagetables: an array of L2 pagetables. Initially, this list is 
    empty. If a L2 pagetable is used for the first time, it is created and the 
    capref is stored for future uses.

	\item Spawninfo: When spawning a new domain, we need to copy the caps for 
    the mappings and the L2 pagetables to the new domain. To be able to reuse 
    our mapping code, we keep a reference to the spawninfo that contains a 
    callback function to be used when mapping for a foreign domain.
\end{itemize}

\subsection{Mapping}

The mapping of a memory frame to the virtual address space of a domain 
consists of the following steps:

\begin{enumerate}
	\item If the address is not user chosen, the information about a free 
    block of virtual addresses is computed from with the information stored in 
    the paging state (see \autoref{ss:mapping_structures}).

	\item The L2 pagetable corresponding to the virtual address to be mapped 
    is read from the L1 pagetable. If this pagetable does not yet exist, a new 
    L2 pagetable is created.

	\item Map the minimum between the number of bytes we have to map and the 
    number of bytes that still fit into the L2 pagetable.

	\item Store the reference to the L2 pagetable and the mapping information 
    to be able to unmap the piece of memory again. 

	\item The steps 2 - 4 are repeated until all memory is mapped. This is the 
    case when the requested size, starting from the virtual address, does not 
    fit into a single L2 pagetable.
\end{enumerate}

\subsection{Unmapping}
Because we stored a fair amount of state, unmapping is easy. All parts of the 
region to unmap are traversed and unmapped. After this is done, the freed 
virtual addresses are added to the list of free vspace again (see 
\autoref{ss:mapping_structures}).
\medskip

One problem we encountered while implementing the unmapping was that it can be 
hard to test or demonstrate. Due to compiler optimizations (especially 
instruction reordering), unmapped memory seemed accessible even after it was 
unmapped for a short time.

