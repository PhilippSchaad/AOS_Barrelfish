\section{Demand Paging}\label{s:paging}

\subsection{Pagefault handling}

A pagefault is triggered when virtual memory is accessed, that has not been 
backed with physical memory. Before handling this exception, we do some sanity 
checks on the accessed address. We check whether the caller tries to 
dereference a NULL pointer, if it is trying to access something in the kernel 
space (above 0x80000000) or if the current stack pointer exceeds the stack. 
Such a behaviour means misbehaviour or a programming error and the calling 
thread is killed.
\medskip

After these checks, the pagefault is handled. We decided to use 
BASE\_PAGE\_SIZE (4kB) sized pieces of memory to benefit from address locality 
and reduce the number of pagefaults. First, the memory is requested from the 
init domain using a non-malloc path (see \autoref{ss:pagefault_ram}). After we 
received the memory, it is mapped to the caller's virtual address space (see 
\autoref{s:mapping}).
\medskip

We did a quick evaluation of other sizes of memory. We run the init domain and 
let it run our test suite and create 5 threads. This resulted in a total of 18 
pagefaults (since the start of the demand paging mechanism). We repeated the 
test but increased the amount of memory that is mapped by a factor of 2. This 
resulted in 12 pagefaults, which means a reduction of 33\%. Because the 
overhead of our pagefault handling is no bottleneck to the system and the 
savings are therefore not that big, we did not implement the larger memory 
size. The increase of the size would generate additional complexity on the 
programming side, as we would have to check first, if the next page is already 
mapped and would need additional testing to make sure that we cover all corner 
cases.
\medskip

The pagefault handler operates on a separate interrupt stack. We encountered an error, in 
which the handling of the pagefault would exceed the initial size of 4kB. 
Increasing the size of the stack to a really large value (16kB) solved that 
problem.

\subsection{RAM alloc in pagefault handling}\label{ss:pagefault_ram}

Our initial pagefault handler occasionally caused pagefaults while handling 
pagefaults, which led to system crashes. We circumvented this behaviour by 
making sure, that the pagefault handler receives a fully stack based (no 
memory allocation) path for requesting a new ram capability. The following 
additional steps were taken, to allow an as direct handling of the pagefault 
as possible:
\begin{itemize}
	\item Prevent scheduling of other threads during pagefault handling

	\item Create a separate lmp\_channel for ram cap requests during pagefaults

	\item Create a separate waitset for the use with the new lmp\_channel, 
    to make sure that we give control to the domain providing ram 
    capabilities. (And back, of course)
\end{itemize}
For more information see section \ref{ss:rpc_ram}.
