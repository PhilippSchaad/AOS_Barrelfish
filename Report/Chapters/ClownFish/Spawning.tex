\section{Spawning Processes}\label{s:spawning-processes}

Our spawning code is nothing fancy and pretty much follows the procedure 
outlined by the book. Let's go over the individual steps.
\medskip

Since we do not have a file-system in our OS, we are dealt a lucky hand in this 
portion. All we have to do to find the correct module is look up the name in 
the multiboot image. Since the call to \texttt{spawn\_load\_by\_name} of our 
spawn library can (but does not have to) contain an arbitrary number of 
arguments, the first thing we do is parse the requested name, filtering out the 
actual program name and separating the arguments away. We now look up that name 
in the multiboot image. Once we have found our module, we identify the frame it 
sits in and map it into our own address space with \texttt{paging\_map\_frame}, 
which gives us a virtual address pointing to our ELF file.
\medskip

In the next step we need to set up a bunch of capabilities for our new process. 
The first one would be an L1 CNode, with and into which we then create a bunch 
of L2 CNodes using \texttt{cnode\_create\_foreign\_l2} (since we have already 
created the L1 node). We map the TASK CNode (which contains information about 
the process) into the child process's capability space, and then back all it's 
L2 CNodes with some (exactly a page) of memory.
\medskip

We now have to set up the child process's virtual address space by creating an 
L1 page-table in our current virtual address space, which we then copy in to 
the child process's L1-page-table slot. This step is followed by calling 
\texttt{paging\_init\_state}, which can now set up the paging state of our new 
process.
\medskip

We are now ready to load the ELF binary and initialize the dispatcher that will 
take care of launching our process. This is a rather tedious process of just 
creating the dispatcher and an endpoint to it and then copying everything into 
the child process's virtual address space. In there, we then fill in the 
dispatcher struct with details like where the GOT sits, what the process's 
domain ID will be initially and what core it will be run on.
\medskip

All that remains to be done now before we invoke the dispatcher, which will 
start our process, is to set its arguments correctly. In this step we will 
check if we have previously received any arguments from the user. If so, we 
will map those into the child process, if not, we will look up if the multiboot 
image provides some arguments and use those instead. After this we are done and 
are ready to run our process by invoking the dispatcher.