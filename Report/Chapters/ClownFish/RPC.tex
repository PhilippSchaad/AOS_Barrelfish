\section{RPC}\label{s:RPC}

For terminology, our actual remote procedure call system is split into two 
sections:
The RPC system, which is build on lmp\_chan's and for transferring messages 
between processes on the same core, and the URPC system, which is build on a 
shared memory page and is for transferring messages between the init processes 
on the two cores.
\medskip

Now, our biggest regret is that we never unified those two systems properly. 
An approach at unification which would have involved rewriting aos\_rpc to 
account for that could have likely provided an uniform interface and saved us 
a lot of pain along the way. But we didn't get there.
\medskip

The development of the RPC system started out as an adhoc implementation of 
just adding one RPC call after another and having all the machinery there 
directly in every single call. This was, naturally, horribly error prone. 
Also, it was based on a missunderstanding of how to approach this - we 
registered a send and recv handler on the channel every time we made one of 
those RPC calls, this turned out to lead to massive issues with multithreading 
and meant processes could only ever receive if they had just sent. We then had 
a single shared recv handler, but the memory RPC call needed its own recv 
handler as that design didn't quite work for that one. "The memory RPC call 
needs to be handled extra" is a common theme that would repeat throughout our 
changes to this system.
\medskip

At first we tried to make some macros and functions while keeping the same 
basic structure, but that too turned out to be really messy and then it 
became time for the grand rewrite to the first version of the current system. 
Instead of init and other processes both doing their own ad-hoc things with 
the RPC setup and calls, we developed a shared framework for it all.
\medskip

This framework is set up such that first we automatically grab a fresh ID to 
mark the logical call we are in so we can associate responses with it, then 
the memory for the logical call gets persisted and enqueued in the list of 
things to send. Further, we register ourselves on the channel for sending if 
nothing is registered there yet.  
Then when it is time to send it automatically gets a standardized encoding 
scheme applied to it (the first 32bit word of every physical message gets 
encoded to contain the type of call, id of the call and length of the total 
payload of the logical message this is part of) and sends it until it is all 
gone or an issue occured.
Further, the framework also has a standardized receive handler which receives 
all calls on that channel and recovers the logical message from the physical 
messages sent, it does so based on the type and id and having a list of which 
unfinished calls it has where it can look up that type and id pair. Then 
finally, when a message is fully reassembled we call a process specific (in 
fact, actually process and channel specific - as we'll see later in the 
detailed API description). Here, an optimisation opportunity for small 
messages - which are the vast majority - presented itself. Instead of 
persisting it all and adding it to the list and all other steps involved, we can do it 
malloc-free by just having that element on the stack and handing it off, as 
we'll return into this function again and in the normal case free the memory.
While this framework now standardized the encoding of messages - and made it 
really convenient to work with rpc calls as we will see in the API description 
- it also presented a large increase in complexity of the code in one 
particular place (yet still better than the thing before, since before we had 
a bunch of places that were not as difficult but still difficult and also very 
labor intensive to keep consistent and we had bug troubles a lot more before 
the rewrite).
The rewrite, however, still had issues. After the first series of minor bugs 
was fixed, we ran into a pretty big issue: The ram RPC handler now ran over 
that system, which used mallocs in various places due to its generality and 
the thus arising need to persist various sizes etc.
Our first and hacky approach was to add special 
conditions for when we were doing the ram\_rpc calls, using buffers of the 
structs involved instead of allocating new memory and therefore taking a slightly special path. 
This, while working for a bit, was naturally doomed to fail as we still got 
pagefaults at times and the pile of patches that held our os together was 
ever increasing.\\
The next step then was to move the ram\_rpc procedure partially out of the system. 
Receiving calls is still part of it, sending is not. This cut down on the amount of memory allocations 
further and we had a buffer for what was still needed, which got us a bit 
further. But, the actual solution to this problem came later however, after it 
had plagued us for weeks with occuring only spuriously and the code around it 
being annoyingly complex and hard to comprehend.
The solution was, that the ram RPC got moved into its own channel, 
independently of the others (as at this time, all RPC calls were involving init
) and used entirely its own machinery - very reminiscent of the original 
implementation. Luckily, on init's side it did not have to be handled 
differently.
However, we still got pagefaults and the occassional strangeness, which 
later got tracked down to a critical bug in the RPC system's sending. It had 
only a single queue for sending instead of a per-channel queue, which meant 
that in particular cases it was possible for messages not to get sent to the 
proper channel or trying to send a message when the channel would not be ready but 
another channel is. This got fixed after a lot of painful investigation as it 
was hard to track down that this caused the errors.
The current RPC subsystem system is split in two parts. In 
aos\_rpc\_shared is (most of) the machinary which was just described.
\medskip

\begin{lstlisting}[caption={RPC system inits}, 
label=lst:rpc_calls_1, numbers=left, stepnumber=1, float, floatplacement=tl, 
frame=tb, language=c]
errval_t init_rpc_server(
             void (*recv_deal_with_msg)(struct recv_list *),
             struct lmp_chan *chan);

errval_t init_rpc_client(
             void (*recv_deal_with_msg)(struct recv_list *),
             struct lmp_chan *chan, 
             struct capref dest);
\end{lstlisting}
\begin{lstlisting}[caption={RPC send}, 
label=lst:rpc_calls_2, numbers=left, stepnumber=1, float, floatplacement=tl, 
frame=tb, language=c]
errval_t send(struct lmp_chan *chan, struct capref cap, 
              unsigned char type, size_t payloadsize, 
              uintptr_t *payload, 
              struct event_closure callback_when_done, 
              unsigned char id);
\end{lstlisting}

The most important calls there are shown in \autoref{lst:rpc_calls_1}.
init\_rpc\_client creates a new lmp\_chan. It 
creates it with an empty remote\_cap, waiting for anyone to connect to it. 
Also, it uses the handler provided as recv\_deal\_with\_msg as the callback to 
use when a logical message is fully reconstructed on this channel. The struct 
recv\_list contains the payload (both the data and a capref if any was sent) 
and its metadata (rpc call type, id, size).
This is logically the same as init\_rpc\_server, except you need to provide 
it a remote cap to connect to.
\medskip

\autoref{lst:rpc_calls_2} depicts the primary way to send messages, it does 
all the explained converting, enqueuing and registering (send\_loop does the 
actual partitioning and sending).
\medskip

Something that at the time seemed useful was having a callback to be called 
once the message was fully sent, but in practice our system never evolved to 
be callback based enough for this to become useful. 
Another feature missing, is a callback to be called in case sending fails, so 
that we can do custom and message specific handling of issues.
\medskip

\begin{lstlisting}[caption={RPC send\_response}, 
label=lst:rpc_calls_3, numbers=left, stepnumber=1, float, floatplacement=tl, 
frame=tb, language=c]
errval_t send_response(struct recv_list *rl, 
                       struct lmp_chan *chan,
                       struct capref cap, 
                       size_t payloadsize, 
                       void *payload);
\end{lstlisting}
\begin{lstlisting}[caption={rpc\_framework}, 
label=lst:rpc_calls_4, numbers=left, stepnumber=1, float, floatplacement=tl, 
frame=tb, language=c]
rpc_framework(void (*inst_recv_handling)(void *arg1, 
                  struct recv_list *data),
              void *recv_handling_arg1,
              unsigned char type,
              struct lmp_chan *chan, 
              struct capref cap, 
              size_t payloadsize, 
              uintptr_t *payload,
              struct event_closure 
                  callback_when_done)
\end{lstlisting}

Finally, \autoref{lst:rpc_calls_3} shows the convenience function 
send\_response, which answers a RPC call with the given cap and data. It does 
a bit of encoding to make sure that the receiver gets the ID of its original 
RPC call in an expected place.
The real gain, however, was not in having those functions, but that they 
allowed us to write the rpc\_framework function (shown in 
\autoref{lst:rpc_calls_4})), which looks really wild - and it is. 
It is in aos\_rpc and works in conjunction with a logical message recv handler 
which is set up in such a way that it takes the type and id info we were given 
and looks it up in the list of calls that we made which we are waiting for an 
answer on.
\medskip

Now, what rpc\_framework does is that in the first two parameters, it takes 
the function which handles the response to the call we are about to make - as 
well as some state for it (arg1 we provide here, data we will get provided). 
And the other parameters are just all the info we need to make a call.
Then, it registers for a response, sends out the message and 
pumps the waitset until we have received an answer (thus being blocking). 
Which means all we now needed to do to add another RPC call was to implement a 
very simple recv handler and 
serialize our data for sending.
\medskip

\begin{lstlisting}[caption={aos\_rpc\_get with helper function}, 
label=lst:rpc_example, numbers=left, stepnumber=1, float, floatplacement=tl, 
frame=tb, language=c]
static void get_nameserver_recv_handler(
    void *arg1, struct recv_list *data)
{
    struct capref *retcap = (struct capref *) arg1;
    *retcap = data->cap;
}

errval_t aos_rpc_get_nameserver(
    struct aos_rpc *rpc, struct capref *retcap)
{
    rpc_framework(get_nameserver_recv_handler,
                  (void *) retcap,
                  RPC_TYPE_GET_NAME_SERVER, 
                  &rpc->chan, NULL_CAP,
                  0, NULL, NULL_EVENT_CLOSURE);
    return SYS_ERR_OK;
}
\end{lstlisting}

\autoref{lst:rpc_example} shows an example for a call for getting the 
nameserver's cap (so we can connect with all the other services by asking the 
nameserver for those).
\medskip

Overall, the RPC system had a long history and features a great API now, that is really simple to use 
and does not need a deep understanding of how lmp\_chan's work. However, this 
was paid for in bugs, complexity and a lot of time. However, in the end, it was a worthwhile 
tradeoff as this setup could rapidly be extend needed.
\medskip

Our one big drawback is that we never made it backend agnostic to integrate 
the cross-core communication system (dubbed URPC) into it.
