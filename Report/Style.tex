\usepackage[T1]{fontenc}
\usepackage[scaled=0.85]{beramono}
\setcounter{secnumdepth}{5}

%add code
\usepackage{listings}
\usepackage{color}
\definecolor{codegreen}{rgb}{0,0.6,0}
\definecolor{codegray}{rgb}{0.5,0.5,0.5}
\definecolor{codepurple}{rgb}{0.58,0,0.82}
 
\lstdefinestyle{mystyle}{
    commentstyle=\color{codegreen},
    keywordstyle=\color{magenta},
    numberstyle=\tiny\color{codegray},
    stringstyle=\color{codepurple},
    basicstyle=\footnotesize,
    breakatwhitespace=false,         
    breaklines=true,                 
    captionpos=b,                    
    keepspaces=true,                 
    numbers=none,                    
    numbersep=5pt,                  
    showspaces=false,                
    showstringspaces=false,
    showtabs=false,                  
    tabsize=2
}

\lstdefinestyle{numbered}{
    commentstyle=\color{codegreen},
    keywordstyle=\color{magenta},
    numberstyle=\tiny\color{codegray},
    stringstyle=\color{codepurple},
    basicstyle=\footnotesize,
    breakatwhitespace=false,         
    breaklines=true,                 
    captionpos=b,                    
    keepspaces=true,                 
    numbers=left,                    
    numbersep=5pt,                  
    showspaces=false,                
    showstringspaces=false,
    showtabs=false,                  
    tabsize=2
}
 
\lstset{style=numbered}

%% to detect ä,ö,ü
\usepackage[utf8]{inputenc}

%% add math symbols
\usepackage{amsmath}
\usepackage{trfsigns}
\usepackage{latexsym}
\usepackage{graphicx}
\usepackage{amssymb}
\usepackage{subcaption}


% used to set maxheight for images
\usepackage[export]{adjustbox}

%add todo
\usepackage{color}
\newcommand{\todo}[1]{\textbf{\color{red}{TODO: #1}}\\}

% make table of contents clickable
\usepackage{hyperref}
\hypersetup{
    colorlinks,
    citecolor=black,
    filecolor=black,
    linkcolor=black,
    urlcolor=black
}

\newcommand{\uU}{\underline{U}}
\newcommand{\uI}{\underline{I}}
\newcommand{\uZ}{\underline{Z}}

% Klammern
\newcommand{\klammern}[1]{\left( #1 \right)}
\newcommand{\eckigeKlammern}[1]{\left[ #1 \right]}
\newcommand{\geschwungeneKlammern}[1]{\left\{ #1 \right\}}

% set image path to the .Grafik folder 
\graphicspath{ {.Grafik/},{Grafik/},{Images/} }
% style von bildern im standartordner
\newcommand{\image}[1]{\includegraphics[max height=100pt,max width=\bildBreite]{#1}}
%define image width
\newcommand{\bildBreite}[0]{\columnwidth}

% Package to include pdf documents
\usepackage[final]{pdfpages}

\usepackage{algorithm}

\newcommand{\function}[1]{\begin{gather*} #1 \end{gather*}}